\documentclass{article}
\usepackage[utf8]{inputenc}
\usepackage[english,russian]{babel}
\usepackage{indentfirst}
\usepackage{misccorr}
\usepackage{graphicx}
\usepackage{amsmath}
\begin{document}
 \begin{table}
 \large
  \vline
  \begin{tabular}{|c c c|}
  \hline 
    \multicolumn{1}{c}{\textbf{Вуз}} \vline
  & \multicolumn{1}{c}{\textbf{Экзаменационный билет}} \vline
  & \multicolumn{1}{c}{\textbf{Начальник}} 
  \vline \\
    \multicolumn{1}{c}{\textbf{Кафедра}} \vline 
  & \multicolumn{1}{c}{\text{по любимому предмету}} \vline
  & \multicolumn{1}{c}{\text{Иванов И.И.}} 
  \vline\\
    \multicolumn{1}{c}{\textit{Ленивых людей}}\vline
  & \multicolumn{1}{c}{\text{\rule{8cm}{0.004cm}}} \vline
  & \multicolumn{1}{c}{} 
  \vline \\
  \hline
  \multicolumn{3}{l}{Вопрос № 1 Решите задачу с массивом и радуйтесь жизни. Выпейте чашку чая.}
  \vline\\
  \cline{1-3}
  \multicolumn{3}{l}{Вопрос № 2}
  \vline\\
  \cline{1-3}
  \multicolumn{3}{l}{Вопрос № 3}
  \vline\\
  \cline{1-3}
  \end{tabular}
  
  
  %\begin{tabular}{|l|}
  %\cline{1-3}
  %\textbf{Вопрос № 1 Решите задачу с массивом и радуйтесь жизни}\\
  %\cline{1-3}
  %\textbf{Вопрос № 2}\\
  %\cline{1-3}
  %\textbf{Вопрос № 3}\\
  %\cline{1-3}
  %\end{tabular} 
  
  %\begin{tabular}[c]{|c|c|c|}
  %\hline
  %\parbox{1.8cm}{Вуз \\ Кафедра} &
  %\parbox{Экзаменационный билет} & 
  %\parbox{2cm}{Начальство}\\
  %\hline
  %\textbf{gdjkhdfjghkfjghfjkghfjghfg} & \textbf{gfkdgldjfgkdfjgkfjgkfjg}\\
  
  %\end{tabular}
 \end{table}
\end{document}
